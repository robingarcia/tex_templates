\documentclass[paper=a5,headsepline=true,9pt,DIV=12,BCOR=1.2cm]{scrbook}
%
% explanation of options: the final version of the thesis is to be
% printed on a5 paper. It can also be printed on a4 paper using the
% «booklet» option. The advantage is, that the thesis will look on
% screen exactly as it will on paper. The thesis can also be printed
% on a5 printers in the copy shop of your liking.
%
% BCOR is the binding margin correction, it will increase the inner
% margin, so that binding can be done. No other geometry information
% is required.
%
% Our current IST binding procedure does not consume the inner margin
% much, so it appears that BCOR should be of size 0. But you might
% want to set it to a nonzero value if you prefer to gently open the
% booklet and not damage the back.

\usepackage{typearea}           % recalculates the type area using the BCOR
                                % argument

\usepackage[utf8]{inputenc}     % text encoding in this file change to
                                % latin1 if you prefer that


\usepackage[sc,osf]{mathpazo}   % the default font used in this
                                % manuscript (a bit thicker than
                                % lmodern), comment this package out
                                % if you want the default font

\usepackage{graphicx}           % to include graphics (pdflatex
                                % includes: png,pdf,jpg and more, but
                                % not eps! [nor ps])
\usepackage{amsmath,amssymb}    % math packages, e.g. for align and pmatrix environments

\usepackage[usenames]{xcolor}   % you might consider using the dvipsnames option for more colors

% some color definitions
\colorlet{lcolor}{blue!40!black}
\colorlet{ucolor}{magenta!40!black}
\colorlet{ccolor}{green!40!black}

\usepackage[colorlinks=true,%
            linkcolor=lcolor,%
            urlcolor=ucolor,%
            citecolor=ccolor]{hyperref} % makes references clickable
            % links will be in specified colors. You may set them to
            % black for final printing.


\usepackage{tikz}               % graphics inside latex; very powerful

\usepackage{microtype} %typographic enhancements

\pagestyle{headings}

