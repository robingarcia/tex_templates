\documentclass[dvips,xcolor=pst]{beamer}

\usetheme{iststyle}
\usepackage[english]{babel}
\usepackage[latin1]{inputenc}
\setbeamercovered{transparent}


\title{Beamer Example for IST style}
\author{Steffen Waldherr}
\institute{
  Institute for Systems Theory and Automatic Control\\
  University of Stuttgart}
\date{February 2007}
\footlinetext{Example talk for IST style}

\begin{document}

\frame[plain]{\titlepage}

\begin{frame}{Using blocks}
\begin{itemize}
\item This slide shows the usage of colored blocks.

\begin{block}{Standard block}
	\begin{itemize}
		\item Nested item environments become smaller.
	\end{itemize}
\end{block}
	
\begin{exampleblock}{Example block}
	\begin{itemize}
		\item Nested item environments become smaller.
	\end{itemize}
\end{exampleblock}
	
\begin{alertblock}{Alert block}
	\begin{itemize}
		\item Nested item environments become smaller.
	\end{itemize}
\end{alertblock}

\begin{block}{}
	\centering Blocks without titles are also useful. 
\end{block}
\end{itemize}
\end{frame}

\frame{
  \frametitle{Now this is a frame with a really long title not fitting on one line}
  
  Note that ist-beamer breaks the long title line and adjust the blue separating line accordingly.
}

\begin{frame}{Using transparent overlays}
\begin{itemize}
\item Shown on the first slide
\pause
\item on the second slide

\begin{block}{}
Blocks can also be covered transparently.
\end{block}
\pause
\item See the latex-beamer documentation for more information on overlays.
\end{itemize}

\end{frame}

\end{document}


